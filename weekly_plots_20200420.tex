\documentclass[landscape]{article}
\usepackage{graphicx}
\usepackage[margin=0.5in]{geometry}

\title{Nationscape Survey Data on COVID-19}
\author{Tyler Reny, PhD Candidate, UCLA\\Ryan Baxter-King, PhD Student, UCLA}
\begin{document}
\maketitle

Nationscape is a survey conducting 500,000 interviews of Americans from July 2019 through December 2020, covering the 2020 campaign and election and providing a nuanced picture of the American electorate. The survey has been in the field since July 10, 2019, and it includes interviews with roughly 6,250 people per week. Nationscape samples are provided by Lucid, a market research platform that runs an online exchange for survey respondents. The samples drawn from this exchange match a set of demographic quotas and are weighted to be representative of the American population. 

Nationscape is a featured project of Democracy Fund Voter Study Group in partnership with UCLA political scientists Lynn Vavreck and Chris Tausanovitch, and USA Today.

\subsection*{Summary of Findings}

\begin{itemize}
	\item Overall, Americans are being exposed, at least second hand, to coronavirus and are taking the pandemic seriously. Over the last four weeks there has been a significant increase in the number of Americans who know someone who has gotten sick (Figure 1). Over that same period, we've seen a significant increase in the percent of all partisans who are taking precautions to avoid getting sick or spreading the virus (Figure 2) with modest partisan gaps between Democrats and Republicans.
	\item There are clear and stable partisan gaps, however, in how worried Americans are about coronavirus with a nearly 20 percentage point gap between Democrats and Republicans (Figure 3). 
	\item These partisan gaps are clear in approval of President Trump's handling of the coronavirus pandemic (Figure 5) with Republicans over 60\% more likely to approve of Trump's performance than Democrats, though this approval has been declining modestly over the last few weeks.
	\item In Figures 6 through 11, we analyze the role that self-interest plays in being concerned about the coronavirus or taking precautions to see whether a) respondents are "exposed" to the virus by being sick of knowing someone who is sick; b) being "unhealthy"; c) or having lower levels of household income increases worry or number of precautions taken. We find:
	\begin{itemize}
		\item Those who are "exposed" are about 8 to 10\% more likely to report being "very concerned" and are slightly more likely to take additional precautions (Figures 6 and 7)
		\item Those who are less healthy are also more likely to be "very concerned" but are not much more likely to take precautions (Figures 8 and 9).
		\item Contrary to expectations, those with the lowest levels of income are less likely report being "very concerned" or take precautions than those in higher income brackets (Figures 10 and 11).
	\end{itemize} 
	\item In Figures 12 and 13 we conduct the same analysis but breaking data out by census region. We find that those in the Northeast have been and continue to be far more likely to be "very concerned" and to take precautions. The rest of the country, however, has been catching up to the Northeast over the last 4 weeks with respect to precautions taken. 
	\item Finally, in Figure 14, we analyze proportions of partisan groups who agree with a number of different conspiracy theories. A notably high percentage of all partisans appear willing to say that these theories are "probably" or "definitely" true.
\end{itemize}

\newpage
\begin{figure}
\begin{center}
\includegraphics[width=\textwidth]{sick.png}
\end{center}
\caption{Exposure To Coronavirus: Points indicate weighted percentage of respondents who indicate that a) they have gotten sick with coronavirus; b) someone in their immediate family has gotten sick with coronavirus; c) someone they work with has gotten sick with coronavirus; or d) someone they know outside of their immediate family or work has gotten sick with coronavirus. Date indicates last field date of survey, each of which is in the field for 1 week. Data from Democracy Fund and UCLA Nationscape.}
\end{figure}

\begin{figure}
\begin{center}
\includegraphics[width=\textwidth]{precautions.png}
\end{center}
\caption{Precautions Taken: Points indicate weighted percentage of respondents who have a) cancelled travel plans for work or pleasure; b) not left the home for a prolonged period of time; c) bought extra groceries or household supplies; d) stopped visiting friends and family; and e) washed their hands more often than they typically do. Date indicates last field date of survey, each of which is in the field for 1 week.}
\end{figure}

\begin{figure}
\begin{center}
\includegraphics[width=0.6\textwidth]{concerned_corona.png}
\end{center}
\caption{How Concerned Are You About Coronavirus: Points indicate weighted percentage of respondents who indicate that they are "very" concerned about coronavirus here in the United States. Date indicates last field date of survey, each of which is in the field for 1 week.}
\end{figure}

\begin{figure}
\begin{center}
\includegraphics[width=0.6\textwidth]{econ_impact_corona.png}
\end{center}
\caption{Economic Impact of Coronavirus: Points indicate weighted percentage of respondents who indicate that their income has been "significantly reduced" or that they have "lost [their] primary source of income" in response to the coronavirus. Date indicates last field date of survey, each of which is in the field for 1 week.}
\end{figure}

\begin{figure}
\begin{center}
\includegraphics[width=0.6\textwidth]{trump_approve_corona.png}
\end{center}
\caption{Trump Approval Handling Coronavirus: Points indicate weighted percentage of respondents who indicate that they either "strongly" or "somewhat" approve of Donald Trump's handling of the coronavirus outbreak. Date indicates last field date of survey, each of which is in the field for 1 week.}
\end{figure}

\begin{figure}
\begin{center}
\includegraphics[width=\textwidth]{personal_impact_sick.png}
\end{center}
\caption{Relationship Between Exposure and Concern: Points indicate weighted percentage of respondents who indicate that they are "very" concerned about coronavirus here in the United States split by whether they have been exposed either personally or second-hand to the coronavirus (they, their family, their coworkers, or other personal contacts have gotten sick). Date indicates last field date of survey, each of which is in the field for 1 week.}
\end{figure}

\begin{figure}
\begin{center}
\includegraphics[width=\textwidth]{personal_impact_sick_count_precautions.png}
\end{center}
\caption{Relationship Between Exposure and Precautions: Points indicate weighted average number of precautions taken from this list: a) cancelled travel plans for work or pleasure; b) not left the home for a prolonged period of time; c) bought extra groceries or household supplies; d) stopped visiting friends and family; and e) washed their hands more often than they typically do. Split by whether they have been exposed either personally or second-hand to the coronavirus (they, their family, their coworkers, or other personal contacts have gotten sick). Date indicates last field date of survey, each of which is in the field for 1 week.}
\end{figure}

\begin{figure}
\begin{center}
\includegraphics[width=\textwidth]{personal_impact.png}
\end{center}
\caption{Relationship Between Health and Concern: Points indicate weighted percentage of respondents who indicate that they are "very" concerned about coronavirus here in the United States split by whether they regularly take prescription medications or not, a rough proxy for health and therefore greater risk from coronavirus. Date indicates last field date of survey, each of which is in the field for 1 week.}
\end{figure}

\begin{figure}
\begin{center}
\includegraphics[width=\textwidth]{personal_impact_count_precautions.png}
\end{center}
\caption{Relationship Between Health and Precautions: Points indicate weighted average number of precautions taken from this list: a) cancelled travel plans for work or pleasure; b) not left the home for a prolonged period of time; c) bought extra groceries or household supplies; d) stopped visiting friends and family; and e) washed their hands more often than they typically do. Split by whether they regularly take prescription medications or not, a rough proxy for health and therefore greater risk from coronavirus. Date indicates last field date of survey, each of which is in the field for 1 week.}
\end{figure}


\begin{figure}
\begin{center}
\includegraphics[width=\textwidth]{personal_impact_income.png}
\end{center}
\caption{Relationship Between Economic Condition and Concern: Points indicate weighted percentage of respondents who indicate that they are "very" concerned about coronavirus here in the United States split out by income terciles. Date indicates last field date of survey, each of which is in the field for 1 week.}
\end{figure}

\begin{figure}
\begin{center}
\includegraphics[width=\textwidth]{personal_impact_income_count_precautions.png}
\end{center}
\caption{Relationship Between Economic Condition and Precautions Taken: Points indicate weighted average number of precautions taken from this list: a) cancelled travel plans for work or pleasure; b) not left the home for a prolonged period of time; c) bought extra groceries or household supplies; d) stopped visiting friends and family; and e) washed their hands more often than they typically do. Split out by income terciles. Date indicates last field date of survey, each of which is in the field for 1 week.}
\end{figure}

\begin{figure}
\begin{center}
\includegraphics[width=0.6\textwidth]{regions_worry.png}
\end{center}
\caption{Regions and Concern: Points indicate weighted percentage of respondents who indicate that they are "very" concerned about coronavirus here in the United States split by census region. Date indicates last field date of survey, each of which is in the field for 1 week.}
\end{figure}

\begin{figure}
\begin{center}
\includegraphics[width=0.6\textwidth]{regions_precautions.png}
\end{center}
\caption{Regions and Precautions Taken: Points indicate weighted average number of precautions taken from this list: a) cancelled travel plans for work or pleasure; b) not left the home for a prolonged period of time; c) bought extra groceries or household supplies; d) stopped visiting friends and family; and e) washed their hands more often than they typically do. Split out by census regions. Date indicates last field date of survey, each of which is in the field for 1 week.}
\end{figure}







\begin{figure}
\begin{center}
\includegraphics[width=0.8\textwidth]{rumors.png}
\end{center}
\caption{Relationship Between Exposure and Concern: Bars indicate weighted percentage of respondents who indicate that it is "definitely" or "probably" true that a) the "U.S is concealing the true scale of coronavirus deaths"; b) "The coronavirus was created in a lab"; c) "The threat of the coronavirus is being exaggerated for political reasons"; d) "A vaccine that prevents infection by the coronavirus already exists and is being withheld from the public"; e) "A treatment that cures people infected by the coronavirus already exists and is being withheld from the public"; f) "People under 30 are less likely to get infected by the coronavirus than older Americans"; and g) "If infected, the coronavirus is no more dangerous than the seasonal flu for people under 30." Date indicates last field date of survey, each of which is in the field for 1 week.}
\end{figure}







\end{document}